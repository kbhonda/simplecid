\documentclass{jarticle}


\usepackage{simplecid}

\AtBeginDocument{%
\special{pdf:mapline scid00-raw Identity-H A-OTF-RyuminPro-Light.otf}
\special{pdf:mapline scid01-raw Identity-H A-OTF-GothicBBBPro-Medium.otf}
}

\sCIDSetting{0}{JY1/mc/m/n}{A-OTF-RyuminPro-Light.otf}
\sCIDSetting{1}{JY1/gt/m/n}{A-OTF-GothicBBBPro-Medium.otf}
\sCIDSetting{1}{JY1/mc/bx/n}{A-OTF-GothicBBBPro-Medium.otf}


\begin{document}



あ\sCID{5421}\sCID{14400}\sCID{14410}

\textbf{あ\sCID{5421}\sCID{14400}\sCID{14410}}


\end{document}


CIDを入力

これを8836(94*94)ごとに区切る
1--8836,8837--17672(=8836*2),...

順番に0群,1群,...とする
入力されたCID番号をcid,
その番号がn群のp番とすると
cid -1 = n * 8836 + R (0<=R<8836)
p = R+1 

1<=p<=8836

R = (ku-1) * 94 + (ten-1) 0<=ten<94

cid => n  (ku,ten)
jis = (ku+0x20) * 0x100 + (ten+0x20) 
